
In C and you want to enter a chroot you can do so directly using the chroot() function:

#include <stdio.h>
#include <unistd.h>

int main(void) {
     FILE *f;

     /* chroot */
     chdir("/tmp");
     if (chroot("/tmp") != 0) {
         perror("chroot /tmp");
         return 1;
     }

     /* do something after chrooting */
     f = fopen("/etc/passwd", "r");
     if (f == NULL) {
         perror("/etc/passwd");
         return 1;
     } else {
         char buf[100];
         while (fgets(buf, sizeof(buf), f)) {
              printf("%s", buf);
         }
     }
     return 0;
}

use chroot inside endpointd.c to safely execute the ./domain executables as needed....

 

simplicity always give birth to flexibility - mo (blockchain ready). simple code is fast code. Unix phiosphy. less is more.

- commands should either throw an exception or raise one or more events.


- Commands are inputs representing instructions to do something.(./domain/domain_constructs as Eric Evans says Translation blunts communication and makes knowledge crunching anemic.)
- Events are outputs representing historical facts of what was done. (/var/log/cityboxio/citybox.events.log)
- Event handlers can listen events in order to issue commands, helping coordinate the different parts of the system. (syslog.conf and syslogd)


Having one command create another command causes ambiguity in the general meaning of commands and events.
If one command "triggers" another command, it is implying that a command is "an historical fact of what was done". This contradicts the intent of these two types of messages, and could become a slippery path, in that other developers could trigger events from events, which could lead to corrupt data and eventual inconsistency.

Rule: commands can't trigger other commands
Rule: commands do somthing and log it after its done as an event.


syslogd.config to monitor eventstream.log and run shellscripts to update the query databases.
BSD syslogd allows log messages to be piped to a program as well as written to one or more files.
implementation of a BSD log monitor often begins with the creation of a program which accepts formatted log messages via
shared stadard input, parses them, and extracts nessesary information.

Packaging citybox as an openbsd package on self-hosted ports server.

log events to program that update database or a program that pass to multiple database update programs.

For maximum performance speed, openbsd supports wirting log messages to an in-memory buffer, 
which allows logging on systems that have no writable disk i.e. RaspberryPi and similar.


A command handler receives a command and brokers a result from the appropriate aggregate. "A result" is either a successful application of the command, or an exception. This is the common sequence of steps a command handler follows:
- Validate the command on its own merits.
- Validate the command on the current state of the aggregate.
- If validation is successful, 0..n events (1 is common).
- Attempt to persist the new events. If there's a concurrency conflict during this step, either give up, or retry things.


Databases are state-only. 
Eventstore is append-only. 
Who should be touched first?
Answer: a "boundedcontext:create_event_command" will result in 
"Eventstore:EventAttempted" -> syslog.conf will execute programs that would Touch the database(s) -> "EventStore:EventConcluded".
/var/log/citybox/opendatahub.events.log
/var/log/citybox/opendatahub.commands.log
/var/log/citybox/opendatahub.queries.log
/var/DATA/citybox/mongodb or postgresql or somthingelse
/var/DATA is an external drive mounted

A good taste in selectin ideas:
- Make(1) used as a code generation tool for different microservices, similar to django/Rails approach for rapid development. 
- KSH scripts and Make(1) used as a DevOps basic tool for different microservices.
- Operating System(OpenBSD) as a development framework for our system is favoured over using a language-specific framework(s). Base tools are favoured over ports. 

Commands and Queries for each microservice are the DSL constructs used in its bounded context.
commands write data to a database. (create_datasource, update_datasource, delete_datasource, etc.)
queries read data from a database. (read_datasource)

- commands write to csv flat-file database or mongodb while recording events to eventstore.
- cqsyncd syncs between commands database and queries database via using eventstore to update queries databases. (how to create a database from events stored in json format on disk as an append-only?)
- queries read from a redis database while recording events to eventstore i.e. RedisQueryAttepmted, RedisQuerySucceeded.

citybox blockchain: use newsyslog(8) to store the rotated "filtered interesting events" opendatahub eventstore of logs into a blockchain(addblock/verify chain) by sending them to a blockchain-node-miniserver.

data-driven architecture: inexpensive commodity hardware (miniservers).

openculture: stackoverflow forum like for urban thingy and IRC for devs along with webinterface.

security between microservices: 
pub/private tokens between microservices.
email-sent access tokens.
signify events.

TODO
- Write Operations - CQRS Commands
-- website sends write events in json to a webserver(obsd httpd) via POST\PUT verbs
-- obsd httpd recevies the json to a specific endpoint 
-- jq is used to make sense of json
-- specific endpoint utility uses endpoint.commands.syslog (and logs via a merkle tree manner of previous syslog messages)
-- every endpoint acts as a single microservers and has its two seperate append-only syslog files(one for commands and one for query events), rotated backup is done via the OS to external drives, snapeshots too. 
-- a filedatabase(csv) daemon that monitors the syslog file uses syslog events to create a state-only file database 
-- a graph database daemon that monitors the syslog file uses syslog events to create a state-only graph database
OR use 	syslog.conf(5) as a unix pub/sub thingy :)
-- each endpoint/microservice stores its events but not "stateful data" 
TODO
- Read Operations - CQRS Query
-- website sends read events in json to a webserver(obsd httpd) via GET verbs
-- obsd httpd recevies the json to a specific endpoint 
-- jq is used to make sense of json
-- specific endpoint utility uses endpoint.query.syslog (and logs via a merkle tree manner of previous syslog messages)

bib:
- www.future-processing.pl/blog/cqrs-simple-architecture/
- blog.codinghorror.com/building-a-computer-the-google-way/
- blog.coddinghorror.com/what-does-stack-overflow-want-to-be-when-it-grows-up/
- udidahan.com/wp-content/uploads/Clarified_CQRS.pdf
- www.usenix.org/legacy/event/sec09/tech/slides/crosby.pdf (Efficent Data structures for tamper-evident logging)kj:w
- www.owasp.org/index.php/Error_Handling,_Auditing_and_Logging#Audit_Trails (secure logging recommendations) 
- https://arxiv.org/pdf/1807.03662.pdf (trialchain: A blockchain-platform for medical data using multichain)
- https://logsentinel.com/white-paper/
- https://openbsd.org/papers/bsdcan-signify.html
- http://www.cqrs.nu/faq (good defeniations)
- https://muut.com/blog/technology/redis-as-primary-datastore-wtf.html
